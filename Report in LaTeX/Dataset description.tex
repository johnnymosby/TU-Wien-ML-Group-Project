\documentclass[11pt]{article}

    \usepackage[breakable]{tcolorbox}
    \usepackage{parskip} % Stop auto-indenting (to mimic markdown behaviour)
    \usepackage{tabularx}
    

    % Basic figure setup, for now with no caption control since it's done
    % automatically by Pandoc (which extracts ![](path) syntax from Markdown).
    \usepackage{graphicx}
    % Maintain compatibility with old templates. Remove in nbconvert 6.0
    \let\Oldincludegraphics\includegraphics
    % Ensure that by default, figures have no caption (until we provide a
    % proper Figure object with a Caption API and a way to capture that
    % in the conversion process - todo).
    \usepackage{caption}
    \DeclareCaptionFormat{nocaption}{}
    \captionsetup{format=nocaption,aboveskip=0pt,belowskip=0pt}

    \usepackage{float}
    \floatplacement{figure}{H} % forces figures to be placed at the correct location
    \usepackage{xcolor} % Allow colors to be defined
    \usepackage{enumerate} % Needed for markdown enumerations to work
    \usepackage{geometry} % Used to adjust the document margins
    \usepackage{amsmath} % Equations
    \usepackage{amssymb} % Equations
    \usepackage{textcomp} % defines textquotesingle
    % Hack from http://tex.stackexchange.com/a/47451/13684:
    \AtBeginDocument{%
        \def\PYZsq{\textquotesingle}% Upright quotes in Pygmentized code
    }
    \usepackage{upquote} % Upright quotes for verbatim code
    \usepackage{eurosym} % defines \euro

    \usepackage{iftex}
    \ifPDFTeX
        \usepackage[T1]{fontenc}
        \IfFileExists{alphabeta.sty}{
              \usepackage{alphabeta}
          }{
              \usepackage[mathletters]{ucs}
              \usepackage[utf8x]{inputenc}
          }
    \else
        \usepackage{fontspec}
        \usepackage{unicode-math}
    \fi

    \usepackage{fancyvrb} % verbatim replacement that allows latex
    \usepackage{grffile} % extends the file name processing of package graphics
                         % to support a larger range
    \makeatletter % fix for old versions of grffile with XeLaTeX
    \@ifpackagelater{grffile}{2019/11/01}
    {
      % Do nothing on new versions
    }
    {
      \def\Gread@@xetex#1{%
        \IfFileExists{"\Gin@base".bb}%
        {\Gread@eps{\Gin@base.bb}}%
        {\Gread@@xetex@aux#1}%
      }
    }
    \makeatother
    \usepackage[Export]{adjustbox} % Used to constrain images to a maximum size
    \adjustboxset{max size={0.9\linewidth}{0.9\paperheight}}

    % The hyperref package gives us a pdf with properly built
    % internal navigation ('pdf bookmarks' for the table of contents,
    % internal cross-reference links, web links for URLs, etc.)
    \usepackage{hyperref}
    % The default LaTeX title has an obnoxious amount of whitespace. By default,
    % titling removes some of it. It also provides customization options.
    \usepackage{titling}
    \usepackage{longtable} % longtable support required by pandoc >1.10
    \usepackage{booktabs}  % table support for pandoc > 1.12.2
    \usepackage{array}     % table support for pandoc >= 2.11.3
    \usepackage{calc}      % table minipage width calculation for pandoc >= 2.11.1
    \usepackage[inline]{enumitem} % IRkernel/repr support (it uses the enumerate* environment)
    \usepackage[normalem]{ulem} % ulem is needed to support strikethroughs (\sout)
                                % normalem makes italics be italics, not underlines
    \usepackage{soul}      % strikethrough (\st) support for pandoc >= 3.0.0
    \usepackage{mathrsfs}
    

    
    % Colors for the hyperref package
    \definecolor{urlcolor}{rgb}{0,.145,.698}
    \definecolor{linkcolor}{rgb}{.71,0.21,0.01}
    \definecolor{citecolor}{rgb}{.12,.54,.11}

    % ANSI colors
    \definecolor{ansi-black}{HTML}{3E424D}
    \definecolor{ansi-black-intense}{HTML}{282C36}
    \definecolor{ansi-red}{HTML}{E75C58}
    \definecolor{ansi-red-intense}{HTML}{B22B31}
    \definecolor{ansi-green}{HTML}{00A250}
    \definecolor{ansi-green-intense}{HTML}{007427}
    \definecolor{ansi-yellow}{HTML}{DDB62B}
    \definecolor{ansi-yellow-intense}{HTML}{B27D12}
    \definecolor{ansi-blue}{HTML}{208FFB}
    \definecolor{ansi-blue-intense}{HTML}{0065CA}
    \definecolor{ansi-magenta}{HTML}{D160C4}
    \definecolor{ansi-magenta-intense}{HTML}{A03196}
    \definecolor{ansi-cyan}{HTML}{60C6C8}
    \definecolor{ansi-cyan-intense}{HTML}{258F8F}
    \definecolor{ansi-white}{HTML}{C5C1B4}
    \definecolor{ansi-white-intense}{HTML}{A1A6B2}
    \definecolor{ansi-default-inverse-fg}{HTML}{FFFFFF}
    \definecolor{ansi-default-inverse-bg}{HTML}{000000}

    % common color for the border for error outputs.
    \definecolor{outerrorbackground}{HTML}{FFDFDF}

    % commands and environments needed by pandoc snippets
    % extracted from the output of `pandoc -s`
    \providecommand{\tightlist}{%
      \setlength{\itemsep}{0pt}\setlength{\parskip}{0pt}}
    \DefineVerbatimEnvironment{Highlighting}{Verbatim}{commandchars=\\\{\}}
    % Add ',fontsize=\small' for more characters per line
    \newenvironment{Shaded}{}{}
    \newcommand{\KeywordTok}[1]{\textcolor[rgb]{0.00,0.44,0.13}{\textbf{{#1}}}}
    \newcommand{\DataTypeTok}[1]{\textcolor[rgb]{0.56,0.13,0.00}{{#1}}}
    \newcommand{\DecValTok}[1]{\textcolor[rgb]{0.25,0.63,0.44}{{#1}}}
    \newcommand{\BaseNTok}[1]{\textcolor[rgb]{0.25,0.63,0.44}{{#1}}}
    \newcommand{\FloatTok}[1]{\textcolor[rgb]{0.25,0.63,0.44}{{#1}}}
    \newcommand{\CharTok}[1]{\textcolor[rgb]{0.25,0.44,0.63}{{#1}}}
    \newcommand{\StringTok}[1]{\textcolor[rgb]{0.25,0.44,0.63}{{#1}}}
    \newcommand{\CommentTok}[1]{\textcolor[rgb]{0.38,0.63,0.69}{\textit{{#1}}}}
    \newcommand{\OtherTok}[1]{\textcolor[rgb]{0.00,0.44,0.13}{{#1}}}
    \newcommand{\AlertTok}[1]{\textcolor[rgb]{1.00,0.00,0.00}{\textbf{{#1}}}}
    \newcommand{\FunctionTok}[1]{\textcolor[rgb]{0.02,0.16,0.49}{{#1}}}
    \newcommand{\RegionMarkerTok}[1]{{#1}}
    \newcommand{\ErrorTok}[1]{\textcolor[rgb]{1.00,0.00,0.00}{\textbf{{#1}}}}
    \newcommand{\NormalTok}[1]{{#1}}

    % Additional commands for more recent versions of Pandoc
    \newcommand{\ConstantTok}[1]{\textcolor[rgb]{0.53,0.00,0.00}{{#1}}}
    \newcommand{\SpecialCharTok}[1]{\textcolor[rgb]{0.25,0.44,0.63}{{#1}}}
    \newcommand{\VerbatimStringTok}[1]{\textcolor[rgb]{0.25,0.44,0.63}{{#1}}}
    \newcommand{\SpecialStringTok}[1]{\textcolor[rgb]{0.73,0.40,0.53}{{#1}}}
    \newcommand{\ImportTok}[1]{{#1}}
    \newcommand{\DocumentationTok}[1]{\textcolor[rgb]{0.73,0.13,0.13}{\textit{{#1}}}}
    \newcommand{\AnnotationTok}[1]{\textcolor[rgb]{0.38,0.63,0.69}{\textbf{\textit{{#1}}}}}
    \newcommand{\CommentVarTok}[1]{\textcolor[rgb]{0.38,0.63,0.69}{\textbf{\textit{{#1}}}}}
    \newcommand{\VariableTok}[1]{\textcolor[rgb]{0.10,0.09,0.49}{{#1}}}
    \newcommand{\ControlFlowTok}[1]{\textcolor[rgb]{0.00,0.44,0.13}{\textbf{{#1}}}}
    \newcommand{\OperatorTok}[1]{\textcolor[rgb]{0.40,0.40,0.40}{{#1}}}
    \newcommand{\BuiltInTok}[1]{{#1}}
    \newcommand{\ExtensionTok}[1]{{#1}}
    \newcommand{\PreprocessorTok}[1]{\textcolor[rgb]{0.74,0.48,0.00}{{#1}}}
    \newcommand{\AttributeTok}[1]{\textcolor[rgb]{0.49,0.56,0.16}{{#1}}}
    \newcommand{\InformationTok}[1]{\textcolor[rgb]{0.38,0.63,0.69}{\textbf{\textit{{#1}}}}}
    \newcommand{\WarningTok}[1]{\textcolor[rgb]{0.38,0.63,0.69}{\textbf{\textit{{#1}}}}}


    % Define a nice break command that doesn't care if a line doesn't already
    % exist.
    \def\br{\hspace*{\fill} \\* }
    % Math Jax compatibility definitions
    \def\gt{>}
    \def\lt{<}
    \let\Oldtex\TeX
    \let\Oldlatex\LaTeX
    \renewcommand{\TeX}{\textrm{\Oldtex}}
    \renewcommand{\LaTeX}{\textrm{\Oldlatex}}
    % Document parameters
    % Document title
    \title{Dataset description}
    
    
    
    
    
    
    
% Pygments definitions
\makeatletter
\def\PY@reset{\let\PY@it=\relax \let\PY@bf=\relax%
    \let\PY@ul=\relax \let\PY@tc=\relax%
    \let\PY@bc=\relax \let\PY@ff=\relax}
\def\PY@tok#1{\csname PY@tok@#1\endcsname}
\def\PY@toks#1+{\ifx\relax#1\empty\else%
    \PY@tok{#1}\expandafter\PY@toks\fi}
\def\PY@do#1{\PY@bc{\PY@tc{\PY@ul{%
    \PY@it{\PY@bf{\PY@ff{#1}}}}}}}
\def\PY#1#2{\PY@reset\PY@toks#1+\relax+\PY@do{#2}}

\@namedef{PY@tok@w}{\def\PY@tc##1{\textcolor[rgb]{0.73,0.73,0.73}{##1}}}
\@namedef{PY@tok@c}{\let\PY@it=\textit\def\PY@tc##1{\textcolor[rgb]{0.24,0.48,0.48}{##1}}}
\@namedef{PY@tok@cp}{\def\PY@tc##1{\textcolor[rgb]{0.61,0.40,0.00}{##1}}}
\@namedef{PY@tok@k}{\let\PY@bf=\textbf\def\PY@tc##1{\textcolor[rgb]{0.00,0.50,0.00}{##1}}}
\@namedef{PY@tok@kp}{\def\PY@tc##1{\textcolor[rgb]{0.00,0.50,0.00}{##1}}}
\@namedef{PY@tok@kt}{\def\PY@tc##1{\textcolor[rgb]{0.69,0.00,0.25}{##1}}}
\@namedef{PY@tok@o}{\def\PY@tc##1{\textcolor[rgb]{0.40,0.40,0.40}{##1}}}
\@namedef{PY@tok@ow}{\let\PY@bf=\textbf\def\PY@tc##1{\textcolor[rgb]{0.67,0.13,1.00}{##1}}}
\@namedef{PY@tok@nb}{\def\PY@tc##1{\textcolor[rgb]{0.00,0.50,0.00}{##1}}}
\@namedef{PY@tok@nf}{\def\PY@tc##1{\textcolor[rgb]{0.00,0.00,1.00}{##1}}}
\@namedef{PY@tok@nc}{\let\PY@bf=\textbf\def\PY@tc##1{\textcolor[rgb]{0.00,0.00,1.00}{##1}}}
\@namedef{PY@tok@nn}{\let\PY@bf=\textbf\def\PY@tc##1{\textcolor[rgb]{0.00,0.00,1.00}{##1}}}
\@namedef{PY@tok@ne}{\let\PY@bf=\textbf\def\PY@tc##1{\textcolor[rgb]{0.80,0.25,0.22}{##1}}}
\@namedef{PY@tok@nv}{\def\PY@tc##1{\textcolor[rgb]{0.10,0.09,0.49}{##1}}}
\@namedef{PY@tok@no}{\def\PY@tc##1{\textcolor[rgb]{0.53,0.00,0.00}{##1}}}
\@namedef{PY@tok@nl}{\def\PY@tc##1{\textcolor[rgb]{0.46,0.46,0.00}{##1}}}
\@namedef{PY@tok@ni}{\let\PY@bf=\textbf\def\PY@tc##1{\textcolor[rgb]{0.44,0.44,0.44}{##1}}}
\@namedef{PY@tok@na}{\def\PY@tc##1{\textcolor[rgb]{0.41,0.47,0.13}{##1}}}
\@namedef{PY@tok@nt}{\let\PY@bf=\textbf\def\PY@tc##1{\textcolor[rgb]{0.00,0.50,0.00}{##1}}}
\@namedef{PY@tok@nd}{\def\PY@tc##1{\textcolor[rgb]{0.67,0.13,1.00}{##1}}}
\@namedef{PY@tok@s}{\def\PY@tc##1{\textcolor[rgb]{0.73,0.13,0.13}{##1}}}
\@namedef{PY@tok@sd}{\let\PY@it=\textit\def\PY@tc##1{\textcolor[rgb]{0.73,0.13,0.13}{##1}}}
\@namedef{PY@tok@si}{\let\PY@bf=\textbf\def\PY@tc##1{\textcolor[rgb]{0.64,0.35,0.47}{##1}}}
\@namedef{PY@tok@se}{\let\PY@bf=\textbf\def\PY@tc##1{\textcolor[rgb]{0.67,0.36,0.12}{##1}}}
\@namedef{PY@tok@sr}{\def\PY@tc##1{\textcolor[rgb]{0.64,0.35,0.47}{##1}}}
\@namedef{PY@tok@ss}{\def\PY@tc##1{\textcolor[rgb]{0.10,0.09,0.49}{##1}}}
\@namedef{PY@tok@sx}{\def\PY@tc##1{\textcolor[rgb]{0.00,0.50,0.00}{##1}}}
\@namedef{PY@tok@m}{\def\PY@tc##1{\textcolor[rgb]{0.40,0.40,0.40}{##1}}}
\@namedef{PY@tok@gh}{\let\PY@bf=\textbf\def\PY@tc##1{\textcolor[rgb]{0.00,0.00,0.50}{##1}}}
\@namedef{PY@tok@gu}{\let\PY@bf=\textbf\def\PY@tc##1{\textcolor[rgb]{0.50,0.00,0.50}{##1}}}
\@namedef{PY@tok@gd}{\def\PY@tc##1{\textcolor[rgb]{0.63,0.00,0.00}{##1}}}
\@namedef{PY@tok@gi}{\def\PY@tc##1{\textcolor[rgb]{0.00,0.52,0.00}{##1}}}
\@namedef{PY@tok@gr}{\def\PY@tc##1{\textcolor[rgb]{0.89,0.00,0.00}{##1}}}
\@namedef{PY@tok@ge}{\let\PY@it=\textit}
\@namedef{PY@tok@gs}{\let\PY@bf=\textbf}
\@namedef{PY@tok@ges}{\let\PY@bf=\textbf\let\PY@it=\textit}
\@namedef{PY@tok@gp}{\let\PY@bf=\textbf\def\PY@tc##1{\textcolor[rgb]{0.00,0.00,0.50}{##1}}}
\@namedef{PY@tok@go}{\def\PY@tc##1{\textcolor[rgb]{0.44,0.44,0.44}{##1}}}
\@namedef{PY@tok@gt}{\def\PY@tc##1{\textcolor[rgb]{0.00,0.27,0.87}{##1}}}
\@namedef{PY@tok@err}{\def\PY@bc##1{{\setlength{\fboxsep}{\string -\fboxrule}\fcolorbox[rgb]{1.00,0.00,0.00}{1,1,1}{\strut ##1}}}}
\@namedef{PY@tok@kc}{\let\PY@bf=\textbf\def\PY@tc##1{\textcolor[rgb]{0.00,0.50,0.00}{##1}}}
\@namedef{PY@tok@kd}{\let\PY@bf=\textbf\def\PY@tc##1{\textcolor[rgb]{0.00,0.50,0.00}{##1}}}
\@namedef{PY@tok@kn}{\let\PY@bf=\textbf\def\PY@tc##1{\textcolor[rgb]{0.00,0.50,0.00}{##1}}}
\@namedef{PY@tok@kr}{\let\PY@bf=\textbf\def\PY@tc##1{\textcolor[rgb]{0.00,0.50,0.00}{##1}}}
\@namedef{PY@tok@bp}{\def\PY@tc##1{\textcolor[rgb]{0.00,0.50,0.00}{##1}}}
\@namedef{PY@tok@fm}{\def\PY@tc##1{\textcolor[rgb]{0.00,0.00,1.00}{##1}}}
\@namedef{PY@tok@vc}{\def\PY@tc##1{\textcolor[rgb]{0.10,0.09,0.49}{##1}}}
\@namedef{PY@tok@vg}{\def\PY@tc##1{\textcolor[rgb]{0.10,0.09,0.49}{##1}}}
\@namedef{PY@tok@vi}{\def\PY@tc##1{\textcolor[rgb]{0.10,0.09,0.49}{##1}}}
\@namedef{PY@tok@vm}{\def\PY@tc##1{\textcolor[rgb]{0.10,0.09,0.49}{##1}}}
\@namedef{PY@tok@sa}{\def\PY@tc##1{\textcolor[rgb]{0.73,0.13,0.13}{##1}}}
\@namedef{PY@tok@sb}{\def\PY@tc##1{\textcolor[rgb]{0.73,0.13,0.13}{##1}}}
\@namedef{PY@tok@sc}{\def\PY@tc##1{\textcolor[rgb]{0.73,0.13,0.13}{##1}}}
\@namedef{PY@tok@dl}{\def\PY@tc##1{\textcolor[rgb]{0.73,0.13,0.13}{##1}}}
\@namedef{PY@tok@s2}{\def\PY@tc##1{\textcolor[rgb]{0.73,0.13,0.13}{##1}}}
\@namedef{PY@tok@sh}{\def\PY@tc##1{\textcolor[rgb]{0.73,0.13,0.13}{##1}}}
\@namedef{PY@tok@s1}{\def\PY@tc##1{\textcolor[rgb]{0.73,0.13,0.13}{##1}}}
\@namedef{PY@tok@mb}{\def\PY@tc##1{\textcolor[rgb]{0.40,0.40,0.40}{##1}}}
\@namedef{PY@tok@mf}{\def\PY@tc##1{\textcolor[rgb]{0.40,0.40,0.40}{##1}}}
\@namedef{PY@tok@mh}{\def\PY@tc##1{\textcolor[rgb]{0.40,0.40,0.40}{##1}}}
\@namedef{PY@tok@mi}{\def\PY@tc##1{\textcolor[rgb]{0.40,0.40,0.40}{##1}}}
\@namedef{PY@tok@il}{\def\PY@tc##1{\textcolor[rgb]{0.40,0.40,0.40}{##1}}}
\@namedef{PY@tok@mo}{\def\PY@tc##1{\textcolor[rgb]{0.40,0.40,0.40}{##1}}}
\@namedef{PY@tok@ch}{\let\PY@it=\textit\def\PY@tc##1{\textcolor[rgb]{0.24,0.48,0.48}{##1}}}
\@namedef{PY@tok@cm}{\let\PY@it=\textit\def\PY@tc##1{\textcolor[rgb]{0.24,0.48,0.48}{##1}}}
\@namedef{PY@tok@cpf}{\let\PY@it=\textit\def\PY@tc##1{\textcolor[rgb]{0.24,0.48,0.48}{##1}}}
\@namedef{PY@tok@c1}{\let\PY@it=\textit\def\PY@tc##1{\textcolor[rgb]{0.24,0.48,0.48}{##1}}}
\@namedef{PY@tok@cs}{\let\PY@it=\textit\def\PY@tc##1{\textcolor[rgb]{0.24,0.48,0.48}{##1}}}

\def\PYZbs{\char`\\}
\def\PYZus{\char`\_}
\def\PYZob{\char`\{}
\def\PYZcb{\char`\}}
\def\PYZca{\char`\^}
\def\PYZam{\char`\&}
\def\PYZlt{\char`\<}
\def\PYZgt{\char`\>}
\def\PYZsh{\char`\#}
\def\PYZpc{\char`\%}
\def\PYZdl{\char`\$}
\def\PYZhy{\char`\-}
\def\PYZsq{\char`\'}
\def\PYZdq{\char`\"}
\def\PYZti{\char`\~}
% for compatibility with earlier versions
\def\PYZat{@}
\def\PYZlb{[}
\def\PYZrb{]}
\makeatother


    % For linebreaks inside Verbatim environment from package fancyvrb.
    \makeatletter
        \newbox\Wrappedcontinuationbox
        \newbox\Wrappedvisiblespacebox
        \newcommand*\Wrappedvisiblespace {\textcolor{red}{\textvisiblespace}}
        \newcommand*\Wrappedcontinuationsymbol {\textcolor{red}{\llap{\tiny$\m@th\hookrightarrow$}}}
        \newcommand*\Wrappedcontinuationindent {3ex }
        \newcommand*\Wrappedafterbreak {\kern\Wrappedcontinuationindent\copy\Wrappedcontinuationbox}
        % Take advantage of the already applied Pygments mark-up to insert
        % potential linebreaks for TeX processing.
        %        {, <, #, %, $, ' and ": go to next line.
        %        _, }, ^, &, >, - and ~: stay at end of broken line.
        % Use of \textquotesingle for straight quote.
        \newcommand*\Wrappedbreaksatspecials {%
            \def\PYGZus{\discretionary{\char`\_}{\Wrappedafterbreak}{\char`\_}}%
            \def\PYGZob{\discretionary{}{\Wrappedafterbreak\char`\{}{\char`\{}}%
            \def\PYGZcb{\discretionary{\char`\}}{\Wrappedafterbreak}{\char`\}}}%
            \def\PYGZca{\discretionary{\char`\^}{\Wrappedafterbreak}{\char`\^}}%
            \def\PYGZam{\discretionary{\char`\&}{\Wrappedafterbreak}{\char`\&}}%
            \def\PYGZlt{\discretionary{}{\Wrappedafterbreak\char`\<}{\char`\<}}%
            \def\PYGZgt{\discretionary{\char`\>}{\Wrappedafterbreak}{\char`\>}}%
            \def\PYGZsh{\discretionary{}{\Wrappedafterbreak\char`\#}{\char`\#}}%
            \def\PYGZpc{\discretionary{}{\Wrappedafterbreak\char`\%}{\char`\%}}%
            \def\PYGZdl{\discretionary{}{\Wrappedafterbreak\char`\$}{\char`\$}}%
            \def\PYGZhy{\discretionary{\char`\-}{\Wrappedafterbreak}{\char`\-}}%
            \def\PYGZsq{\discretionary{}{\Wrappedafterbreak\textquotesingle}{\textquotesingle}}%
            \def\PYGZdq{\discretionary{}{\Wrappedafterbreak\char`\"}{\char`\"}}%
            \def\PYGZti{\discretionary{\char`\~}{\Wrappedafterbreak}{\char`\~}}%
        }
        % Some characters . , ; ? ! / are not pygmentized.
        % This macro makes them "active" and they will insert potential linebreaks
        \newcommand*\Wrappedbreaksatpunct {%
            \lccode`\~`\.\lowercase{\def~}{\discretionary{\hbox{\char`\.}}{\Wrappedafterbreak}{\hbox{\char`\.}}}%
            \lccode`\~`\,\lowercase{\def~}{\discretionary{\hbox{\char`\,}}{\Wrappedafterbreak}{\hbox{\char`\,}}}%
            \lccode`\~`\;\lowercase{\def~}{\discretionary{\hbox{\char`\;}}{\Wrappedafterbreak}{\hbox{\char`\;}}}%
            \lccode`\~`\:\lowercase{\def~}{\discretionary{\hbox{\char`\:}}{\Wrappedafterbreak}{\hbox{\char`\:}}}%
            \lccode`\~`\?\lowercase{\def~}{\discretionary{\hbox{\char`\?}}{\Wrappedafterbreak}{\hbox{\char`\?}}}%
            \lccode`\~`\!\lowercase{\def~}{\discretionary{\hbox{\char`\!}}{\Wrappedafterbreak}{\hbox{\char`\!}}}%
            \lccode`\~`\/\lowercase{\def~}{\discretionary{\hbox{\char`\/}}{\Wrappedafterbreak}{\hbox{\char`\/}}}%
            \catcode`\.\active
            \catcode`\,\active
            \catcode`\;\active
            \catcode`\:\active
            \catcode`\?\active
            \catcode`\!\active
            \catcode`\/\active
            \lccode`\~`\~
        }
    \makeatother

    \let\OriginalVerbatim=\Verbatim
    \makeatletter
    \renewcommand{\Verbatim}[1][1]{%
        %\parskip\z@skip
        \sbox\Wrappedcontinuationbox {\Wrappedcontinuationsymbol}%
        \sbox\Wrappedvisiblespacebox {\FV@SetupFont\Wrappedvisiblespace}%
        \def\FancyVerbFormatLine ##1{\hsize\linewidth
            \vtop{\raggedright\hyphenpenalty\z@\exhyphenpenalty\z@
                \doublehyphendemerits\z@\finalhyphendemerits\z@
                \strut ##1\strut}%
        }%
        % If the linebreak is at a space, the latter will be displayed as visible
        % space at end of first line, and a continuation symbol starts next line.
        % Stretch/shrink are however usually zero for typewriter font.
        \def\FV@Space {%
            \nobreak\hskip\z@ plus\fontdimen3\font minus\fontdimen4\font
            \discretionary{\copy\Wrappedvisiblespacebox}{\Wrappedafterbreak}
            {\kern\fontdimen2\font}%
        }%

        % Allow breaks at special characters using \PYG... macros.
        \Wrappedbreaksatspecials
        % Breaks at punctuation characters . , ; ? ! and / need catcode=\active
        \OriginalVerbatim[#1,codes*=\Wrappedbreaksatpunct]%
    }
    \makeatother

    % Exact colors from NB
    \definecolor{incolor}{HTML}{303F9F}
    \definecolor{outcolor}{HTML}{D84315}
    \definecolor{cellborder}{HTML}{CFCFCF}
    \definecolor{cellbackground}{HTML}{F7F7F7}

    % prompt
    \makeatletter
    \newcommand{\boxspacing}{\kern\kvtcb@left@rule\kern\kvtcb@boxsep}
    \makeatother
    \newcommand{\prompt}[4]{
        {\ttfamily\llap{{\color{#2}[#3]:\hspace{3pt}#4}}\vspace{-\baselineskip}}
    }
    

    
    % Prevent overflowing lines due to hard-to-break entities
    \sloppy
    % Setup hyperref package
    \hypersetup{
      breaklinks=true,  % so long urls are correctly broken across lines
      colorlinks=true,
      urlcolor=urlcolor,
      linkcolor=linkcolor,
      citecolor=citecolor,
      }
    % Slightly bigger margins than the latex defaults
    
    \geometry{verbose,tmargin=1in,bmargin=1in,lmargin=1in,rmargin=1in}
    
    

\begin{document}
    
    \maketitle
    
    

    
    \section{Dataset Description Report}\label{dataset-description-report}

\subsection{Group Members}\label{group-members}

\begin{itemize}
\tightlist
\item
  Ruslan Basyrov (12209898)
\item
  Iana Bembeeva (52017248)
\item
  Ekaterina Grigorashchenko (12432933)
\end{itemize}

\subsection{Introduction}\label{introduction}

This report provides a detailed description of two datasets used for
analysis: a weather dataset predicting rainfall and a salary dataset.
Each dataset's characteristics, attributes, and significance are
discussed.

    \subsection{Dataset 1: Weather Observations
Dataset}\label{dataset-1-weather-observations-dataset}
\raggedright
\subsubsection{Context}\label{context}

The first dataset comprises daily weather observations collected over
approximately ten years from various locations across Australia. This
dataset enables the prediction of next-day rain, answering the crucial
question of whether to carry an umbrella.
\href{https://www.kaggle.com/datasets/jsphyg/weather-dataset-rattle-package}{Link
to Kaggle}.

\subsubsection{Characteristics}\label{characteristics}

\begin{itemize}
\tightlist
\item
  \textbf{Samples}: Approximately 10 years of daily observations (exact
  count of samples may vary).
\item
  \textbf{Attributes}: 22 attributes, including both numerical and
  categorical types.
\end{itemize}

\subsubsection{Attribute Types and Unique
Values}\label{attribute-types-and-unique-values}

\begin{longtable}[]{@{}
  >{\raggedright\arraybackslash}p{(\linewidth - 8\tabcolsep) * \real{0.1626}}
  >{\raggedright\arraybackslash}p{(\linewidth - 8\tabcolsep) * \real{0.0813}}
  >{\raggedright\arraybackslash}p{(\linewidth - 8\tabcolsep) * \real{0.1220}}
  >{\raggedright\arraybackslash}p{(\linewidth - 8\tabcolsep) * \real{0.1301}}
  >{\raggedright\arraybackslash}p{(\linewidth - 8\tabcolsep) * \real{0.5041}}@{}}
\toprule\noalign{}
\begin{minipage}[b]{\linewidth}\raggedright
Attribute
\end{minipage} & \begin{minipage}[b]{\linewidth}\raggedright
Type
\end{minipage} & \begin{minipage}[b]{\linewidth}\raggedright
Unique Values
\end{minipage} & \begin{minipage}[b]{\linewidth}\raggedright
Missing Values
\end{minipage} & \begin{minipage}[b]{\linewidth}\raggedright
Description
\end{minipage} \\
\midrule\noalign{}
\endhead
\bottomrule\noalign{}
\endlastfoot
Date & Ordinal & 3436 & 0 & The date of observation. \\
Location & Nominal & 49 & 0 & The name of the weather station
location. \\
MinTemp & Ratio & 389 & 1485 & Minimum temperature in degrees
Celsius. \\
MaxTemp & Ratio & 505 & 1261 & Maximum temperature in degrees
Celsius. \\
Rainfall & Ratio & 681 & 3261 & Amount of rainfall recorded for the day
in mm. \\
Evaporation & Ratio & 358 & 62790 & Class A pan evaporation (mm) in the
24 hours to 9am. \\
Sunshine & Ratio & 145 & 69835 & Number of hours of bright sunshine in
the day. \\
WindGustDir & Nominal & 16 & 10326 & Direction of the strongest wind
gust in the last 24 hours. \\
WindGustSpeed & Ratio & 67 & 10263 & Speed (km/h) of the strongest wind
gust in the last 24 hours. \\
WindDir9am & Nominal & 16 & 10566 & Direction of the wind at 9am. \\
WindDir3pm & Nominal & 16 & 4228 & Direction of the wind at 3pm. \\
WindSpeed9am & Ratio & 43 & 1767 & Wind speed (km/h) averaged over 10
minutes prior to 9am. \\
WindSpeed3pm & Ratio & 44 & 3062 & Wind speed (km/h) averaged over 10
minutes prior to 3pm. \\
Humidity9am & Ratio & 101 & 2654 & Humidity (percent) at 9am. \\
Humidity3pm & Ratio & 101 & 4507 & Humidity (percent) at 3pm. \\
Pressure9am & Ratio & 546 & 15065 & Atmospheric pressure (hPa) at
9am. \\
Pressure3pm & Ratio & 549 & 15028 & Atmospheric pressure (hPa) at
3pm. \\
Cloud9am & Ratio & 10 & 55888 & Fraction of sky obscured by cloud at 9am
(in oktas). \\
Cloud3pm & Ratio & 10 & 1767 & Fraction of sky obscured by cloud at 3pm
(in oktas). \\
Temp9am & Ratio & 441 & 3609 & Temperature (degrees C) at 9am. \\
Temp3pm & Ratio & 502 & 1485 & Temperature (degrees C) at 3pm. \\
RainToday & Nominal & 2 & 3261 & Yes if precipitation (mm) exceeds 1mm,
otherwise No. \\
RainTomorrow & Nominal & 2 & 3267 & Indicates whether it will rain
tomorrow (Yes/No). \\
\end{longtable}

\subsubsection{Comments on Reformatting}\label{comments-on-reformatting}

\begin{itemize}
\tightlist
\item
  \textbf{Date}: Should be reformatted to \texttt{datetime} for better
  handling in time series analysis.
\end{itemize}

\subsubsection{Target Attribute}\label{target-attribute}

\begin{itemize}
\tightlist
\item
  \textbf{RainTomorrow}: This binary attribute indicates whether it will
  rain the following day (Yes/No). It is crucial for predicting weather
  patterns and making informed decisions.
\end{itemize}

\subsubsection{Importance of Dataset}\label{importance-of-dataset}

Understanding the distribution of values in attributes helps in
preprocessing steps, such as handling missing values and feature
selection, to build accurate predictive models.
 
            
    
    \centering{\begin{tabular}{llllll}
\toprule
 & 0 & 1 & 2 & 3 & 4 \\
\midrule
Date & 2008-12-01 & 2008-12-02 & 2008-12-03 & 2008-12-04 & 2008-12-05 \\
Location & Albury & Albury & Albury & Albury & Albury \\
MinTemp & 13.400000 & 7.400000 & 12.900000 & 9.200000 & 17.500000 \\
MaxTemp & 22.900000 & 25.100000 & 25.700000 & 28.000000 & 32.300000 \\
Rainfall & 0.600000 & 0.000000 & 0.000000 & 0.000000 & 1.000000 \\
Evaporation & NaN & NaN & NaN & NaN & NaN \\
Sunshine & NaN & NaN & NaN & NaN & NaN \\
WindGustDir & W & WNW & WSW & NE & W \\
WindGustSpeed & 44.000000 & 44.000000 & 46.000000 & 24.000000 & 41.000000 \\
WindDir9am & W & NNW & W & SE & ENE \\
WindDir3pm & WNW & WSW & WSW & E & NW \\
WindSpeed9am & 20.000000 & 4.000000 & 19.000000 & 11.000000 & 7.000000 \\
WindSpeed3pm & 24.000000 & 22.000000 & 26.000000 & 9.000000 & 20.000000 \\
Humidity9am & 71.000000 & 44.000000 & 38.000000 & 45.000000 & 82.000000 \\
Humidity3pm & 22.000000 & 25.000000 & 30.000000 & 16.000000 & 33.000000 \\
Pressure9am & 1007.700000 & 1010.600000 & 1007.600000 & 1017.600000 & 1010.800000 \\
Pressure3pm & 1007.100000 & 1007.800000 & 1008.700000 & 1012.800000 & 1006.000000 \\
Cloud9am & 8.000000 & NaN & NaN & NaN & 7.000000 \\
Cloud3pm & NaN & NaN & 2.000000 & NaN & 8.000000 \\
Temp9am & 16.900000 & 17.200000 & 21.000000 & 18.100000 & 17.800000 \\
Temp3pm & 21.800000 & 24.300000 & 23.200000 & 26.500000 & 29.700000 \\
RainToday & No & No & No & No & No \\
RainTomorrow & No & No & No & No & No \\
\bottomrule
\end{tabular}
}

    

    \subsubsection{Dataset visualizations}\label{dataset-visualizations}

\begin{enumerate}
\def\labelenumi{\arabic{enumi}.}
\tightlist
\item
  \textbf{Distribution of RainTomorrow} - understanding the balance
  between classes
\end{enumerate}

    \begin{enumerate}
\def\labelenumi{\arabic{enumi}.}
\setcounter{enumi}{1}
\tightlist
\item
  \textbf{Histograms of Numeric Attributes}
\end{enumerate}

    \begin{center}
    \adjustimage{max size={0.9\linewidth}{0.9\paperheight}}{Dataset description_files/Dataset description_7_0.png}
    \end{center}
    { \hspace*{\fill} \\}
    
    \subsection{Dataset 2: Employee Salary
Dataset}\label{dataset-2-employee-salary-dataset}
\raggedright
\subsubsection{Context}\label{context}

The second dataset provides annual salary information, including gross
and overtime pay, for all active, permanent employees of Montgomery
County, MD, for the calendar year 2016. This data is essential for
analyzing salary distribution and identifying trends in employee
compensation.
\href{https://www.openml.org/search?type=data&id=42125&sort=runs&status=active}{Link
to OpenML}

\subsubsection{Characteristics}\label{characteristics}

\begin{itemize}
\tightlist
\item
  \textbf{Samples}: 9222 employee records
\item
  \textbf{Attributes}: 13 attributes with a mix of numeric and
  categorical types.
\end{itemize}

\subsubsection{Attribute Types and Unique
Values}\label{attribute-types-and-unique-values}

\begin{tabularx}{\linewidth}{@{}
    >{\raggedright\arraybackslash}p{5cm} % Attribute
    >{\raggedright\arraybackslash}p{2cm} % Type
    >{\raggedright\arraybackslash}p{1.5cm} % Unique Values
    >{\raggedright\arraybackslash}X % Description (auto-expanding)
    @{}}
\toprule
\textbf{Attribute} & \textbf{Type} & \textbf{Unique Values} & \textbf{Description} \\
\midrule
full\_name & Nominal & 9222 & Employee's full name. \\
gender & Nominal & 2 & Employee's gender (F, M). \\
current\_annual\_salary & Ratio & 3403 & Employee's current annual salary in USD. \\
2016\_gross\_pay\_received & Ratio & 8977 & Total gross pay received in 2016. \\
2016\_overtime\_pay & Ratio & 6176 & Total overtime pay received in 2016. \\
department & Nominal & 37 & Department of the employee. \\
department\_name & Nominal & 37 & Full name of the department. \\
division & Nominal & 694 & Division of the employee within the department. \\
assignment\_category & Nominal & 2 & Employment category (Fulltime-Regular, Parttime-Regular). \\
employee\_position\_title & Nominal & 385 & Job title of the employee. \\
underfilled\_job\_title & Nominal & 84 & Job title of the position being underfilled, if applicable. \\
date\_first\_hired & Ordinal & 2264 & Date the employee was first hired. \\
year\_first\_hired & Interval & 51 & Year the employee was first hired. \\
\bottomrule
\end{tabularx}

\subsubsection{Target Attribute}\label{target-attribute}

\begin{itemize}
\tightlist
\item
  \textbf{current\_annual\_salary}: The primary target attribute used
  for salary analysis and modeling. Understanding its distribution is
  crucial for various analyses, including equity and budgeting.
\end{itemize}

\subsubsection{Comments on Reformatting}\label{comments-on-reformatting}

\begin{itemize}
\tightlist
\item
  \textbf{date\_first\_hired}: Should be reformatted to
  \texttt{datetime} for better handling in time analysis.
\end{itemize}

\subsubsection{Importance of Dataset}\label{importance-of-dataset}

The distribution of numeric values in the salary dataset provides
insight into compensation trends, while the categorical data (such as
gender and department) allows for analyzing disparities and ensuring
equitable pay practices.

    
\begin{table}[h]
    \centering
    \begin{tabular}{@{}p{5cm} p{3cm} p{3cm} p{3cm}@{}}
    \toprule
    \textbf{Attribute} & \textbf{Instance 1} & \textbf{Instance 2} & \textbf{Instance 3} \\
    \midrule
    full\_name & Aarhus, Pam J. & Aaron, David J. & Aaron, Marsha M. \\
    gender & F & M & F \\
    current\_annual\_salary & 69,222.18 & 97,392.47 & 104,717.28 \\
    2016\_gross\_pay\_received & 71,225.98 & 103,088.48 & 107,000.24 \\
    2016\_overtime\_pay & 416.1 & 3,326.19 & 1,353.32 \\
    department & POL & POL & HHS \\
    department\_name & Department of Police & Department of Police & Department of Health\ldots \\
    division & MSB Information Mgmt\ldots & ISB Major Crimes Divi\ldots & Adult Protective and\ldots \\
    assignment\_category & Fulltime-Regular & Fulltime-Regular & Fulltime-Regular \\
    employee\_position\_title & Office Services Coord\ldots & Master Police Officer & Social Worker IV \\
    underfilled\_job\_title & None & None & None \\
    date\_first\_hired & 09/22/1986 & 09/12/1988 & 11/19/1989 \\
    year\_first\_hired & 1986 & 1988 & 1989 \\
    \bottomrule
    \end{tabular}
    \caption{Example of the Data Instance}
    \label{tab:data_instance}
    \end{table}

    
    \subsubsection{Dataset visualizations}\label{dataset-visualizations}

\begin{enumerate}
\def\labelenumi{\arabic{enumi}.}
\tightlist
\item
  \textbf{Distribution of Current Annual Salary} - analyzing salary
  ranges and identifying any outliers
\end{enumerate}

    \begin{center}
    \adjustimage{max size={0.75\linewidth}{0.75\paperheight}}{Dataset description_files/Dataset description_12_0.png}
    \end{center}
    { \hspace*{\fill} \\}
    
    \begin{center}
    \adjustimage{max size={0.75\linewidth}{0.75\paperheight}}{Dataset description_files/Dataset description_13_0.png}
    \end{center}
    { \hspace*{\fill} \\}
    
    \begin{center}
    \adjustimage{max size={0.75\linewidth}{0.75\paperheight}}{Dataset description_files/Dataset description_14_0.png}
    \end{center}
    { \hspace*{\fill} \\}
    
    \begin{center}
    \adjustimage{max size={0.75\linewidth}{0.75\paperheight}}{Dataset description_files/Dataset description_15_0.png}
    \end{center}
    { \hspace*{\fill} \\}
    
    \begin{center}
    \adjustimage{max size={0.75\linewidth}{0.75\paperheight}}{Dataset description_files/Dataset description_16_0.png}
    \end{center}
    { \hspace*{\fill} \\}
    
    \begin{center}
    \adjustimage{max size={0.75\linewidth}{0.75\paperheight}}{Dataset description_files/Dataset description_17_0.png}
    \end{center}
    { \hspace*{\fill} \\}
    
    \begin{enumerate}
\def\labelenumi{\arabic{enumi}.}
\setcounter{enumi}{1}
\tightlist
\item
  \textbf{Histograms of Numeric Attributes}
\end{enumerate}



    \begin{center}
    \adjustimage{max size={0.75\linewidth}{0.75\paperheight}}{Dataset description_files/Dataset description_19_1.png}
    \end{center}
    { \hspace*{\fill} \\}
    

    \begin{center}
    \adjustimage{max size={0.75\linewidth}{0.75\paperheight}}{Dataset description_files/Dataset description_20_1.png}
    \end{center}
    { \hspace*{\fill} \\}
    


    \begin{center}
    \adjustimage{max size={0.75\linewidth}{0.75\paperheight}}{Dataset description_files/Dataset description_21_1.png}
    \end{center}
    { \hspace*{\fill} \\}
    

\subsubsection{Conclusion}\label{conclusion}

Both datasets offer valuable insights into weather prediction and
employee compensation. By understanding their characteristics,
distributions, and the significance of attributes, we can apply
appropriate data preprocessing techniques and build effective predictive
models.


    % Add a bibliography block to the postdoc
    
    
    
\end{document}
